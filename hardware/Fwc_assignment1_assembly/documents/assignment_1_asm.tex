\documentclass[10pt,a4paper,twocolumn]{report}
\usepackage[latin1]{inputenc}
\usepackage{amsmath}
\usepackage{amsfonts}
\usepackage{amssymb}
\usepackage{graphicx}
\usepackage{tikz}
\usepackage{tabularx}
\usepackage{lipsum}
\usetikzlibrary{matrix,calc}
\usepackage[margin=0.5in]{geometry}
\usepackage{tikz}
\usetikzlibrary{arrows,shapes.gates.logic.US,shapes.gates.logic.IEC,calc}
\usepackage{listings}

\begin{document}

\lstset{
frame=single, 
breaklines=true,
columns=fullflexible
}
\centering \textbf {\underline{ASSIGNMENT-1}}\\
\vspace{5mm}
\raggedright \textbf{Name} :\hspace{1mm} \textbf{Mannava Venkatasai} \\
\textbf{Roll}\hspace{3mm}   : \textbf{FWC22030} \\
\textbf{Email}\hspace{3mm} : \textbf{venkatasaimannava9948@gmail.com} \vspace{7mm} \\

\raggedright \textbf{PROBLEM STATEMENT:}\vspace{2mm}
\raggedright \\Draw the Logic Circuit for the following Boolean Expression :
\textbf{f(x,y,z,w) = ($x'$+y).z + $w'$}
\vspace{1cm}
\\ \textbf {Logic circuit:} \\
\vspace{10mm}
\tikzstyle{branch}=[fill,shape=circle,minimum size=1pt,inner sep=0pt]
\begin{tikzpicture}[label distance=1mm]

    \node (x3) at (0,0) {$x$};
    \node (x2) at (1,0) {$y$};
    \node (x1) at (2,0) {$z$};
    \node (x0) at (3,0) {$w$};

    \node[not gate US, draw, rotate=-90] at ($(x3)+(0,-1)$) (Not3) {};
    \node[not gate US, draw, rotate=-90] at ($(x0)+(0,-1)$) (Not0) {};

    \node[or gate US, draw, logic gate inputs=nn] at ($(x0)+(1,-2)$) (Or1) {};

    \node[and gate US, draw, logic gate inputs=nn, anchor=input 1] at ($(Or1.output)+(1,-1)$) (And1) {};
    \node[or gate US, draw, logic gate inputs=nn, anchor=input 1] at ($(Or1.output -| And1.output)+(1,-1.5)$) (Or2) {};

    \foreach \i in {3,0}
    {
        \path (x\i) -- coordinate (punt\i) (x\i |- Not\i.input);
        \draw (punt\i) node[branch] {} -| (Not\i.input);
    }
    \draw (x2) |- (Or1.input 2);
    \draw (Not3.output) |- (Or1.input 1);
    \draw (Or1.output) -- ([xshift=0.2cm]Or1.output) |- (And1.input 1);
    \draw (x1) |- (And1.input 2);
    \draw (And1.output) -- ([xshift=0.2cm]And1.output) |- (Or2.input 1);
    \draw (Not0.output) |- (Or2.input 2);
    \draw (Or2.output) -- ([xshift=0.5cm]Or2.output) node[above] {$f_1$};
   
    
  
\end{tikzpicture}
\vspace{5mm}
\\ \raggedright \textbf{\underline{AIM:}}\vspace{2mm}
\\ \raggedright To Draw the Logic Circuit and implement using Arduino for the following Boolean Expression :
\\ F(x,y,z,w) = ($x'$+y).z + $w'$
\vspace{5mm}
\\ \raggedright \textbf{\underline{Components:}}\vspace{2mm}
\begin{table}[ht]
\centering % used for centering table
\begin{tabular}{c c c} % centered columns (4 columns)
\hline\hline %inserts double horizontal lines
S.No & Component & Number \\ [0.5ex] % inserts table 
\hline
1 & Arduino & 1 \\
2 & Bread Board & 1 \\
3 & Jumer Wires(M-M) & 10 \\
4 & 7447 IC & 1 \\
5 & Seven segment display & 1 \\ [1ex] 
\hline
\end{tabular}
\end{table}
\vspace{5mm}
\\ \raggedright \textbf{\underline{Procedure:}}\vspace{4mm}
\\ \raggedright 1) First make the 2,3,4,5 digital pins of arduino as input pins and declare                                the 13 pin as output pin.
\vspace{1mm}
\\ 2) Write the given logic in code and upload in to the arduino.
\vspace{1mm}
\\ 3) Connect the output pin i.e pin 13 of arduino to the one of the input of 7447 IC i.e pin A and the remaining input pins(pins:D,B,C) are connected to ground.
\vspace{1mm}
\\ 4) Connect the outputs of IC 7447 i.e a,b,c,d,e,f,g,h to the corresponding pins of sevensegment display.
\vspace{1mm}
\\ 5) The out put will be displayed in seven segment display either 1 or 0 corresponds to the out given boolean expression.
\vspace{10mm}
\\ \raggedright \textbf{\underline{OUTPUTS:}}\vspace{7mm}
\\ \raggedright \textbf{\underline{Truthtable:}}\vspace{2mm}
   \begin{center}
\begin{tabularx}{0.4\textwidth} { 
  | >{\centering\arraybackslash}X 
  | >{\centering\arraybackslash}X 
  | >{\centering\arraybackslash}X
  | >{\centering\arraybackslash}X
  | >{\centering\arraybackslash}X | }
\hline
\textbf{x} &\textbf{y} & \textbf{z} & \textbf{w} & \textbf{f} \\
\hline
0 & 0 & 0 & 0 & 1 \\  
\hline
0 & 0 & 0 & 1 & 0 \\ 
\hline
0 & 0 & 1 & 0 & 1 \\
\hline
0 & 0 & 1 & 1 & 1 \\
\hline
0 & 1 & 0 & 0 & 1 \\  
\hline
0 & 1 & 0 & 1 & 0 \\ 
\hline
0 & 1 & 1 & 0 & 1 \\
\hline
0 & 1 & 1 & 1 & 1 \\
\hline
1 & 0 & 0 & 0 & 1 \\
\hline
1 & 0 & 0 & 1 & 0 \\
\hline
1 & 0 & 1 & 0 & 1 \\
\hline
1 & 0 & 1 & 1 & 0 \\
\hline
1 & 1 & 0 & 0 & 1 \\
\hline
1 & 1 & 0 & 1 & 0 \\
\hline
1 & 1 & 1 & 0 & 1 \\
\hline
1 & 1 & 1 & 1 & 1 \\
\hline
\end{tabularx}
\end{center}
\raggedright \begin{center} \includegraphics[scale=0.17]{output.jpeg} \end{center} \begin{center} The output is displayed as 0 in seven segment display corresponds to the given inputs. \end{center}
\vspace{5mm}
\raggedright \begin{center} \includegraphics[scale=0.17]{out.jpeg} \end{center}
\begin{center} The output is displayed as 1 in seven segment display corresponds to the given inputs. \end{center}
\vspace{5mm}
\raggedright \textbf{\underline{Conclusion:}}\vspace{7mm}
\\ Hence I have drawn the logic circuit for the given logic expression and I have implemented the circuit in arduino and verified the outputs.
\vspace{10mm}
\\ \raggedright Code is awailable in the following directory \\
\begin{lstlisting}
https://github.com/Mannava123455/Mannava-Venkatasai/blob/main/Fwc_assignment1_assembly/assignment_1.asm
\end{lstlisting}
\end{document}